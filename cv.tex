%%%%%%%%%%%%%%%%%%%%%%%%%%%%%%%%%%%%%%%%%
% Medium Length Professional CV
% LaTeX Template
% Version 2.0 (8/5/13)
%
% This template has been downloaded from:
% http://www.LaTeXTemplates.com
%
% Original author:
% Trey Hunner (http://www.treyhunner.com/)
%
% Important note:
% This template requires the resume.cls file to be in the same directory as the
% .tex file. The resume.cls file provides the resume style used for structuring the
% document.
%
%%%%%%%%%%%%%%%%%%%%%%%%%%%%%%%%%%%%%%%%%

%----------------------------------------------------------------------------------------
%	PACKAGES AND OTHER DOCUMENT CONFIGURATIONS
%----------------------------------------------------------------------------------------

\documentclass{resume} % Use the custom resume.cls style
\usepackage{hyperref, textcomp, lipsum}
\usepackage{enumitem}
\usepackage[dvipsnames]{xcolor}
\hypersetup{
    colorlinks=true,
    linkcolor=blue,
    filecolor=blue,      
    urlcolor=blue,
}
\usepackage[left=0.5in,top=0.5in,right=0.6in,bottom=0.5in]{geometry} % 

\name{Kalpesh Krishna}
\address{\textit{\href{mailto:kalpesh@cs.umass.edu}{kalpesh@cs.umass.edu}} \\ \href{https://www.linkedin.com/in/kalpesh-krishna-6b3827a6/}{LinkedIn} \\ \href{https://github.com/martiansideofthemoon}{Github} \\ \href{https://twitter.com/kalpeshk2011}{Twitter} \\ \href{https://scholar.google.com/citations?user=9g2BsMUAAAAJ&hl=en&authuser=1}{Google Scholar} \\ \href{https://www.semanticscholar.org/author/Kalpesh-Krishna/26161085}{Semantic Scholar}}
\address{\textbf{WEBPAGE}: \textit{\href{https://martiansideofthemoon.github.io}{https://martiansideofthemoon.github.io}}}

\begin{document}
%\footnotetext[1]{Use the URL \href{http://martiansideofthemoon.github.io}{martiansideofthemoon.github.io} in case {\color{blue} hyperlinks} do not work}
%----------------------------------------------------------------------------------------
%	EDUCATION SECTION
%----------------------------------------------------------------------------------------
\begin{rSection}{Education}
\vspace*{0.1in}
{\bf University of Massachusetts, Amherst} \hfill {Sept '18 - Present} \\ MS/PhD in Computer Science \\ (advised by \textit{\href{https://people.cs.umass.edu/~miyyer/}{Prof. Mohit Iyyer}}) \\
Major GPA: 4.0/4.0\\\\
{\bf Indian Institute of Technology, Bombay} \hfill {July '14 - July '18} \\ 
B.Tech in Electrical Engineering, Minor in Computer Science\\
(advised by \textit{\href{https://www.cse.iitb.ac.in/~pjyothi/}{Prof. Preethi Jyothi}})\\
Major GPA: 9.74/10 (\textit{$\mathbf{2^{nd}}$ among 66 students}), Minor GPA: 10/10
\end{rSection}
%----------------------------------------------------------------------------------------
%	WORK EXPERIENCE SECTION
%----------------------------------------------------------------------------------------

\begin{rSection}{Internships}
\vspace*{0.1in}
{\bf Google Research India} { \hfill June '21 - September '21}\\ \textit{Research Intern under \href{http://talukdar.net/}{Partha Talukdar} and \href{https://sites.google.com/view/bidisha-samanta/}{Bidisha Samanta}} {\hfill Bangalore, India (remote)}

{\bf Google Brain} { \hfill May '20 - August '20}\\ \textit{Research Intern under \href{https://sites.google.com/site/royaurko/}{Aurko Roy}}{\hfill Mountain View, CA (remote)}

{\bf Google AI Language}{ \hfill May '19 - September '19}\\ \textit{Research Intern under \href{https://ai.google/research/people/GauravSinghTomar/}{Gaurav Singh Tomar} and \href{www.ankurparikh.com}{Ankur Parikh}}{\hfill New York, NY}

{\bf Toyota Technological Institute at Chicago}{ \hfill May '17 - July '17}\\ \textit{Research Intern under \href{http://ttic.uchicago.edu/~klivescu/}{Karen Livescu}, \href{http://ttic.uchicago.edu/~llu/}{Liang Lu} and \href{http://ttic.uchicago.edu/~kgimpel/}{Kevin Gimpel}}{\hfill Chicago, IL}

{\bf Mozilla Foundation}{\hfill May '16 - August '16} \\ \textit{Google Summer of Code Intern under \href{https://github.com/armenzg}{Armen Zambrano}}{ \hfill Mumbai, India / London, UK}
\end{rSection}

\begin{rSection}{Awards \& Scholarships}
\vspace*{0.1in}
\begin{itemize}[leftmargin=*]
\item \textbf{Google PhD Fellowship 2021}
\item PhD Candidacy with Distinction at UMass Amherst (Fall 2020)
\item Outstanding Reviewer at ACL 2020, ICLR 2021  (Top 7-12\% reviewers)
\item CICS Graduate Fellowship 2018, Victor Lesser Graduate Scholarship 2019 at UMass Amherst
\item \textbf{Sharad Maloo Memorial Gold Medal} at IIT Bombay Convocation 2018\\
\textit{(For outstanding academic and extra-curricular achievements)}
\item \textbf{Cargill Global Leaders Scholarship 2016-18} \\
\textit{(Awarded by the \href{https://en.wikipedia.org/wiki/Institute_of_International_Education}{Institute of International Education} and \href{https://en.wikipedia.org/wiki/Cargill}{Cargill} for academic and extra-curricular achievements and leadership potential)}
\item Institute Organizational Color 2016-17 at IIT Bombay \\
\textit{(For achievements while leading the Web and Coding Club)}
\item Institute Academic Prize 2015-16 at IIT Bombay
\end{itemize}
\end{rSection}

\pagebreak

\begin{rSection}{Selected Papers}
\vspace*{0.1in}
\begin{itemize}[leftmargin=*]
\item \href{https://arxiv.org/abs/2103.06332}{Hurdles to Progress in Long-form Question Answering} \\
\textit{Kalpesh Krishna}, Aurko Roy, Mohit Iyyer \\
\textbf{NAACL 2021}
\item \href{https://arxiv.org/abs/2010.05700}{Reformulating Unsupervised Style Transfer as Paraphrase Generation} \\ \textit{Kalpesh Krishna}, John Wieting, Mohit Iyyer \\ \textbf{EMNLP 2020} 
\item \href{https://arxiv.org/abs/1910.12366}{Thieves on Sesame Street! Model Extraction of BERT-based APIs} \\ \textit{Kalpesh Krishna}, Gaurav S. Tomar, Ankur P. Parikh, Nicolas Papernot, Mohit Iyyer \\ \textbf{ICLR 2020}
\item \href{https://arxiv.org/abs/1906.02622}{Generating Question-Answer Hierarchies} \\ \textit{Kalpesh Krishna}, Mohit Iyyer \\ \textbf{ACL 2019}
\end{itemize}
\end{rSection}

\begin{rSection}{Other Papers}
\vspace*{0.1in}
\begin{itemize}[leftmargin=*]
\item \href{https://arxiv.org/abs/2109.09115}{Do Long-Range Language Models Actually Use Long-Range Context?} \\
Simeng Sun, \textit{Kalpesh Krishna}, Andrew Mattarella-Micke and Mohit Iyyer \\
\textbf{EMNLP 2021}

\item \href{https://arxiv.org/abs/2103.02537}{Weakly-Supervised Open-Retrieval Conversational Question Answering} \\
Chen Qu, Liu Yang, Cen Chen, W. Bruce Croft, \textit{Kalpesh Krishna} and Mohit Iyyer \\
\textbf{ECIR 2021}
\item \href{https://arxiv.org/abs/2103.00751}{Long Document Summarization in a Low Resource Setting using Pretrained Language Models} \\ Ahsaas Bajaj, Pavitra Dangati, \textit{Kalpesh Krishna}, Pradhiksha Kumar, Rheeya Uppaal, Bradford Windsor, Eliot Brenner, Dominic Dotterrer, Rajarshi Das, Andrew McCallum \\
\textbf{ACL Student Research Workshop 2021}
\item \href{https://arxiv.org/abs/2102.03718}{An Analysis of Frame-skipping in Reinforcement Learning} \\ Shivaram Kalyanakrishnan, Siddharth Aravindan, Vishwajeet Bagdawat, Varun Bhatt, Harshith Goka, Archit Gupta, \textit{Kalpesh Krishna}, Vihari Piratla \\
\textbf{arXiv 2021}
\item \href{https://joss.theoj.org/papers/10.21105/joss.01832}{SunPy: A Python package for Solar Physics} \\ Stuart J. Mumford and \textit{others} \\ \textbf{JOSS 2020}

\item \href{https://arxiv.org/abs/1906.02780}{Syntactically Supervised Transformers for Faster Neural Machine Translation} \\ Nader Akoury, \textit{Kalpesh Krishna}, Mohit Iyyer \\ \textbf{ACL 2019}
\item \href{https://arxiv.org/abs/1903.07820}{Trick or TReAT: Thematic Reinforcement for Artistic Typography} \\ Purva Tendulkar, \textit{Kalpesh Krishna}, Ramprasaath R. Selvaraju, Devi Parikh \\ \textbf{ICCC 2019} ({\textit{Best Presentation Award}})
\item \href{https://arxiv.org/abs/1808.07733}{Revisiting the Importance of Encoding Logic Rules in Sentiment Classification} \\ \textit{Kalpesh Krishna}, Preethi Jyothi, Mohit Iyyer \\ \textbf{EMNLP 2018} ({\textit{oral presentation}}, \textit{short paper})
\item \href{https://arxiv.org/abs/1807.06234}{Hierarchical Multitask Learning for CTC-based Speech Recognition} \\ \textit{Kalpesh Krishna}, Shubham Toshniwal, Karen Livescu \\
\textbf{arXiv 2018}
\item \href{https://arxiv.org/abs/1710.10398}{A Study of All-Convolutional Encoders for Connectionist Temporal Classification}\\ \textit{Kalpesh Krishna}, Liang Lu, Kevin Gimpel,  Karen Livescu\\ \textbf{ICASSP 2018} ({ \textit{Awarded SPS Travel Grant}})
%
\end{itemize}

\textbf{Other Projects}: \url{https://martiansideofthemoon.github.io/other_projects/}
\end{rSection}

%\begin{rSection}{Software Papers}
%\vspace*{0.1in}
%\begin{itemize}[leftmargin=*]

%\end{itemize}
%\end{rSection}

\begin{rSection}{Invited Talks}
\vspace*{0.1in}
\begin{itemize}[leftmargin=*]
\item University of Southern California, July 2021 \\
\textit{(On progress in text generation \& perils of its evaluation)}
\item University of Texas at Austin, June 2021 \\
\textit{(On progress in text generation \& perils of its evaluation)}
\item Google Research, May 2021\\
\textit{(On long-form question answering)}
\item IBM Research, April 2020\\
\textit{(On model extraction attacks on BERT-based APIs)}
\item AllenNLP Summit at AI2, August 2019 \\
\textit{(Lightning talks on using AllenNLP for \href{https://github.com/martiansideofthemoon/allennlp-probe-hw}{education})}
\item UMass Data Science Research Symposium, April 2019 \\
\textit{(On hierarchical question-answer generation)}
\item IIT Bombay Computer Science Cybersecurity Club Seminar, April 2018 \\
\textit{(On adversarial examples for machine learning)}
\item IIT Bombay BioBytes Seminar / WnCC Reflections Seminar, February 2018\\
\textit{(On end-to-end speech recognition)}
\end{itemize}
\end{rSection}

\begin{rSection}{Responsibilities}
\vspace*{0.1in}
\begin{itemize}[leftmargin=*]
\item \textbf{Co-organizing} a weekly talk series ``Machine Learning and Friends Lunch'' at UMass Amherst
\item \textbf{Mentoring} three UMass undergraduates on their honors thesis (2019-2021) and student teams for the industrial mentorship projects (COMPSCI 696) in Spring 2019, 2020, 2021
\item \textbf{Graduate Teaching Assistant} at UMass Amherst for \textit{Deep Learning for NLP} (Spring 2019) and \textit{Advanced Natural Language Processing} (Fall 2020)
\item \textbf{Institute Student Mentor} at IIT Bombay (2017-18)
\item \textbf{Manager, Web and Coding Club} at IIT Bombay (2016-17)
\item \textbf{Teaching Assistant} at IIT Bombay in \textit{Computer Programming} (2016) and \textit{Linear Algebra} (2017)
\item \textbf{Web Coordinator} for Mood Indigo 2016 at IIT Bombay
\end{itemize}
\end{rSection}

\begin{rSection}{Academic Service}
\vspace*{0.1in}
\begin{itemize}[leftmargin=*]
\item \textbf{Program Committee / Reviewer} for EMNLP '19-'21; ACL '20-'21; ICLR '21-'22; NAACL '21; NeurIPS '21; IJCV '19 and several other workshops
\item \textbf{Secondary Reviewer} for ACL '19, NAACL '19, ICLR '19
\item \textbf{Student Volunteer} at ACL '19 (awarded \href{http://www.acl2019.org/EN/student-scholarship-applications-volunteers.xhtml}{ACL 2019 Student Scholarship}), AKBC '19, ICLR '20
\end{itemize}
\end{rSection}

\pagebreak

\begin{rSection}{Other Achievements}
\begin{itemize}[leftmargin=*]
\item Selected for \href{https://www.lti.cs.cmu.edu/2017-jsalt-undergraduate}{JSALT '17}, organized by JHU's \href{https://www.clsp.jhu.edu/}{Center for Language and Speech Processing}\footnotemark[2] \footnotetext[2]{Couldn't attend due to clashing college schedule}
\item \textbf{Top 10} at the Astronomy Olympiad's Indian Selection Camp for IOAA '14, (\textit{20000 candidates})
\item All India Rank 93 in \href{https://en.wikipedia.org/wiki/Joint_Entrance_Examination}{JEE Advanced} '14 \textit{(out of 126,000 candidates)} \\ All India Rank 34 in \href{https://en.wikipedia.org/wiki/Joint_Entrance_Examination}{JEE Mains} '14 \textit{(out of 1,400,000 candidates)} \\
All India Top 100 in \href{https://en.wikipedia.org/wiki/Central_Board_of_Secondary_Education}{CBSE XII} '14 \textit{(out of 1,000,000 candidates)} \\
All India Rank 2 in \href{https://en.wikipedia.org/wiki/Indian_Certificate_of_Secondary_Education}{ICSE} '12 \textit{(out of 132,000 candidates)}\\
\textit{Awarded by Pune city institutions for being the city topper in each of these exams. Featured in the Pune city editions of several newspapers including The Times of India and The Indian Express.}
% Awarded AP grade (Top 1% of class) inComputer Programming,Basic BiologyandData Analysis.
\item Passed the \href{https://www.iapt.org.in/exams/nse/nsep.html}{National Standard Exam in Physics} '14 and \href{https://www.iapt.org.in/exams/nse/nsec-b-a-js.html}{National Standard Exam in Chemistry} '14, the first rounds in the Indian selection procedure for the International Physics Olympiad and the International Chemistry Olympiad respectively
\item Selected for the \href{http://www.kvpy.iisc.ernet.in/main/index.htm}{Kishore Vaigyanik Protsahan Yojana} (KVPY) Award '14 (1,000/20,000 applicants)
%\item Times of India NIE Student of the Year '11
\end{itemize}
\end{rSection}

\begin{rSection}{Extra-Curricular}
\begin{itemize}[leftmargin=*]
\item \textbf{Blogging} - Active \href{http://martiansideofthemoon.github.io/archive.html}{blogger} on MS/PhD applications, biking, bird photography and computer science
\item Led the development of a \href{https://www.wncc-iitb.org/wiki/index.php/The_Web_and_Coding_Club}{beginner's programming wiki} with the Web and Coding Club, IIT Bombay
\item \textbf{StackOverflow} - Active \href{https://stackoverflow.com/users/5080995/martianwars}{contributor} in 2016-17, top 7\% overall
\item \textbf{Open Source} - Actively \href{http://martiansideofthemoon.github.io/open_source/}{contributed} to Mozilla in 2015-16 including \href{https://wiki.mozilla.org/Auto-tools/New_Contributor/Quarter_of_Contribution}{QoC 2015}, \href{https://summerofcode.withgoogle.com/archive/2016/projects/}{GSoC 2016}
\item \textbf{Karate} - Black Belt (1st Dan) trained in Kyokushin Kai for seven years, district level winner
\item \textbf{Abacus \& Mental Arithmetic} - \href{https://alohamindmath.com/}{Aloha} grandmaster, national and state level winner
\end{itemize}
\textbf{Hobbies} --- \href{https://martiansideofthemoon.github.io/birding/}{birdwatching}, photography, hiking, biking, reading, star gazing
\end{rSection}
%----------------------------------------------------------------------------------------

\end{document}
